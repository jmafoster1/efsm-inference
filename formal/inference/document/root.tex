\documentclass[10pt,DIV16,a4paper,abstract=true,twoside=semi,openright]{scrreprt}
\usepackage[USenglish]{babel}
\usepackage[numbers, sort&compress]{natbib}
\usepackage{isabelle,isabellesym}
\usepackage{booktabs}
\usepackage{paralist}
\usepackage{graphicx}
\usepackage{amssymb}
\usepackage{xspace}
\usepackage{xcolor}
\usepackage{hyperref}
\usepackage{rotating}


\pagestyle{headings}
\isabellestyle{default}
\setcounter{tocdepth}{1}
\newcommand{\ie}{i.\,e.\xspace}
\newcommand{\eg}{e.\,g.\xspace}
\newcommand{\thy}{\isabellecontext}
\renewcommand{\isamarkupsection}[1]{%
  \begingroup%
  \def\isacharunderscore{\textunderscore}%
  \section{#1 (\thy)}%
  \def\isacharunderscore{-}%
  \expandafter\label{sec:\isabellecontext}%
  \endgroup%
}

\newcommand{\orcidID}[1]{} % temp. hack

\newcommand{\repeatisanl}[1]
{\ifnum#1=0\else\isanewline\repeatisanl{\numexpr#1-1}\fi}
\newcommand{\snip}[4]{\repeatisanl#2#4\repeatisanl#3}

\title{A Formal Model of Extended Finite State Machine}%
\author{Michael~Foster\orcidID{0000-0001-8233-9873} \and
 Ramsay~G.~Taylor\orcidID{0000-0002-4036-7590} \and
 Achim~D.~Brucker\orcidID{0000-0002-6355-1200} \and
 John~Derrick\orcidID{0000-0002-6631-8914}}
\publishers{
  Department of Computer Science\\
  The University of Sheffield\\
  Sheffield, UK\\
  \texttt{\{%
	\href{mailto:jmafoster1@sheffield.ac.uk}{jmafoster1},
	\href{mailto:a.brucker@sheffield.ac.uk}{a.brucker},
	\href{mailto:r.g.taylor@sheffield.ac.uk}{r.g.taylor},
	\href{mailto:j.derrick@sheffield.ac.uk}{j.derrick}
  \}@sheffield.ac.uk}
}

\begin{document}
\maketitle
\begin{abstract}
  In this AFP entry, we provide a formal implementation of a state-merging technique to infer extended finite state machines (EFSMs), complete with output and update functions, from black-box traces. In particular, we define the  We also define the \emph{subsumption in context} relation as a means of determining whether one transition is able to account for the behaviour of another. Building on this, we define the \emph{direct subsumption} relation, which lifts the \emph{subsumption in context} relation to EFSM level such that we can use it to determine whether it is safe to merge a given pair of transitions. We also provide a number of different \emph{heuristics} which can be used to abstract away concrete values into \emph{registers} so that more states and transitions can be merged. We also provide proofs of the various conditions which must hold in order for the transitions introduced by these heuristics to subsume (that is \emph{account for the behaviour of}) their ungeneralised counterparts. A Code Generator setup to create executable Scala code is also defined.
  \begin{quote}
    \bigskip
    \noindent{\textbf{Keywords:} EFSMs, Model inference, Reverse engineering }
  \end{quote}
\end{abstract}


\tableofcontents
\cleardoublepage

\chapter{Introduction}\label{chap:intro}
This AFP entry provides a formal implementation of a state-merging technique to infer EFSMs from black-box traces and is an accompaniment to \cite{foster2019}. The technique builds off classical FSM inference techniques which work by merging states which share behaviour. The process first builds a Prefix Tree Acceptor from the traces, and then states are iteratively merged to form a smaller model.

Most notably, we formalise the definition of \emph{direct subsumption.} When merging EFSM transitions, one must \emph{account for} the behaviour of the other. The \emph{subsumption in context} relation from \cite{foster2018} formalises the intuition that, in certain contexts, a transition $t_2$ reproduces the behaviour of, and updates the data state in a manner consistent with, another transition $t_1$, meaning that $t_2$ can be used in place of $t_1$ with no observable difference in behaviour. This relation requires us to supply a context in which to test subsumption, but there is a problem when we try to apply this to inference: Which context should we use? The \emph{directly subsumes} relation presented here incorporates subsumption into a relation which can be used to determine if it is safe to merge a pair of transitions, and was one of the main contributions of \cite{foster2019}. It is this which allows us to take the subsumption relation from \cite{foster2018} and use it in the inference process.

The rest of this document is automatically generated from the formalization in Isabelle/HOL, i.e., all content is checked by Isabelle.  Overall, the structure of this document follows the theory dependencies (see \autoref{fig:session-graph}).

\begin{sidewaysfigure}
  \centering
  \resizebox{\textheight}{!}{\includegraphics[height=\textheight]{session_graph}}
  \caption{The Dependency Graph of the Isabelle Theories.\label{fig:session-graph}}
\end{sidewaysfigure}

\nocite{foster.ea:efsm:2018}

\clearpage

\input{session}


{\small
  \bibliographystyle{abbrvnat}
  \bibliography{root}
}
\end{document}
\endinput

%%% Local Variables:
%%% mode: latex
%%% TeX-master: t
%%% End:
